\documentclass[12pt,a4paper]{article}

\usepackage[pdftex]{graphicx}
\usepackage{amsmath}
\usepackage{amsopn,amscd,amsthm,amssymb,xypic,rotating}
\usepackage{thmtools}
\usepackage[utf8]{inputenc}
\usepackage{changebar}
%\usepackage[dvinames]{xcolor}
\usepackage{mathtools}

\input{macros.tex}

\title{BiRating - Iterative averaging on a bipartite graph of Beat Saber player skills and map difficulties}
\author{Juan Casanova}

\begin{document}

\maketitle

Abstract here

\section{Introduction}

Unsupervised learning, iterative algorithms in graphs. Differences with this.

Beat Saber, maps, players, scores, BeatLeader. Previous star algorithms. Differences with this (e.g. supervised vs unsupervised).

\section{Algorithm description}

Basic principles, what the algorithm does, and why.
Talk about the error here.

Subsection on convergence.

\section{Data preparation and implementation}

Implementation details, data preparation, what data I have, what did I do.
Talk about hyperparameter search

\section{Results}

Actual results, direct discussion and its meaning. Talk about convergence again.

Evaluation of results compared to star rating algorithm and community discussion.

\section{Limitations}

Inherent algorithm limitations: Convergence.

Data limitations: Data quality, need scores, abuse, not suitable for live system, which scores to pick, irregularity of scores.

Application limitations: Multidimensionality of skill and difficulty, curve shape matters.

\section{Related and future work}

ALS and other more advanced algorithms.

Mention existing approaches again.

Pattern recognition with neural networks: Supervised data issue, and what is authoritative. Feature selection. Data preprocessing. Curve shape estimation. 

\section{Conclusion}

Interesting approach. Why it is interesting compared to other approaches. Works to some degree. Limitations. More work needed. Helped identify some core difficulties with the approach.

\end{document}